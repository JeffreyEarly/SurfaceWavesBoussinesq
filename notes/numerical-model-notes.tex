\documentclass[11pt]{article}
\usepackage{geometry}                % See geometry.pdf to learn the layout options. There are lots.
\geometry{letterpaper}                   % ... or a4paper or a5paper or ... 
%\geometry{landscape}                % Activate for for rotated page geometry
%\usepackage[parfill]{parskip}    % Activate to begin paragraphs with an empty line rather than an indent
\usepackage{graphicx}
\usepackage{amssymb}
\usepackage{epstopdf}
\usepackage{amsmath}
\DeclareGraphicsRule{.tif}{png}{.png}{`convert #1 `dirname #1`/`basename #1 .tif`.png}

\DeclareMathOperator{\diag}{diag}

\title{Shoaling Waves Model}
\author{Jeffrey J. Early}
%\date{}                                           % Activate to display a given date or no date

\begin{document}
\maketitle
\section{Equations}
We start with the Boussinesq equations from Nwogu, 1993. In his paper these are equations (25) and can be written as,
\begin{align}
\eta_t + \left( (h+\eta)u\right)_x + \partial_x \left[ \left( \frac{z^2}{2} - \frac{h^2}{6}\right) h u_{xx} + \left(z+\frac{h}{2}\right) h (hu)_{xx} \right] &=0 \\
u_t + g \eta_x + u u_x + \left[ \frac{z^2}{2} u_{xxt} + z (hu_t)_{xx} \right] &= 0
\end{align}
where $z$ is the depth of integration and can be set to some function of $h$, the bathymetry. Let's set $z=-\beta h$, so that, after some rearrangement,
\begin{align}
\eta_t + \left( (h+\eta)u\right)_x + \partial_x \left[ \left( \frac{\beta^2}{2} - \beta + \frac{1}{3} \right) h^3 u_{xx} + \left(-\beta +\frac{1}{2}\right) h^2h_{xx} u  \right] &=0 \\
u_t + g \eta_x + u u_x + \left[ \left( \frac{\beta^2}{2} - \beta \right) h^2 u_{xxt} - \beta h h_{xx} u_t \right] &= 0.
\end{align}
In his paper Nwogu uses the parameter $\alpha$ instead of $\beta$, the relationship between the two parameters is $\alpha = \frac{\beta^2}{2} - \beta$ and $\beta = 1 - \sqrt{1+2\alpha}$. Note that when $u$ is found at the bottom of the ocean, this requires that $\beta =1$, and $\alpha=-\frac{1}{2}$. If $u$ is found at the surface, then $\beta=0$ and $\alpha=0$.

One of the key points of Nwogu's paper was that an `optimal' value $\alpha$ can be chosen in which the linear frequency dispersion relationship most closely matches the true frequency dispersion relationship (for both the phase and group speed). The linear dispersion relation is,
\begin{equation}
\frac{\omega^2}{k^2} = g h \left[ \frac{ 1 - \left(\alpha + \frac{1}{3} \right) (kh)^2}{1 - \alpha(kh)^2} \right].
\end{equation}
Nwogu find the optimal choice of $\alpha$ to be $\alpha = -0.393$, corresponding to $\beta = 0.537$ in our notation.

\section{Model}

For our model, we need to write the equations in the form $\frac{dy}{dt} = F$. So, rearranging we find that,
\begin{align}
\eta_t  &=-\left( (h+\eta)u\right)_x - \partial_x \left[ \left( \alpha + \frac{1}{3} \right) h^3 u_{xx} + \left(-\beta +\frac{1}{2}\right) h^2h_{xx} u  \right] \\
\left[ 1+ \alpha h^2 \partial_{xx} - \beta h h_{xx} \right]u_t  &= - g \eta_x - u u_x.
\end{align}
where $\alpha$ and $\beta$ are being intermingled freely to make the notation more compact. We can continue to write this in a more computationally efficient form, by reducing the number of transformations that have to be made.
\begin{align}
\eta_t  &=- \partial_x \left[\left(h +  \left(-\beta +\frac{1}{2}\right) h^2h_{xx} +\eta  \right)u +  \left( \alpha + \frac{1}{3} \right) h^3 u_{xx}   \right] \\
\left[ 1+ \alpha h^2 \partial_{xx} - \beta h h_{xx} \right]u_t  &= - g \eta_x - u u_x.
\end{align}

The tricky part is the operator,
\begin{equation}
L = 1+ \alpha h^2 \partial_{xx} - \beta h h_{xx}
\end{equation}
which must be inverted at each time step. Using $C$ to denote the forward cosine transform (or transform to whatever basis desired), and $D_{xx}$ to denote the differentiation matrix in that basis, this becomes,
\begin{align}
L &= I + \diag{ \left( \alpha h^2 \right)} C^{-1} D_{xx} C - \diag{ \left( \beta h h_{xx} \right)}.
\end{align}
where the operator $\diag{a(x_i)}$ indicates a matrix with the the coefficients of $a$ along the diagonal.


\end{document}  